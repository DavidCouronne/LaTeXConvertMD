\documentclass[10pt,a4paper]{article}
\usepackage[utf8]{inputenc}
\usepackage{fourier}
\usepackage[scaled=0.875]{helvet}
\renewcommand{\ttdefault}{lmtt}
\usepackage{amsmath,amssymb,makeidx}
\usepackage{fancybox}
\usepackage{diagbox}
\usepackage{tabularx}
\usepackage[normalem]{ulem}
\usepackage{pifont}
\usepackage{textcomp}
\usepackage{graphicx}
\usepackage{multicol} 
\newcommand{\euro}{\eurologo{}}
%Tapuscrit : François Hache
\usepackage{pst-all}
\newcommand{\R}{\textbf{R}}
\newcommand{\N}{\mathbb{N}}
\newcommand{\D}{\mathbb{D}}
\newcommand{\Z}{\mathbb{Z}}
\newcommand{\Q}{\mathbb{Q}}
\newcommand{\C}{\mathbb{C}}
\usepackage[left=3.5cm, right=3.5cm, top=3cm, bottom=3cm]{geometry}
\newcommand{\vect}[1]{\overrightarrow{\,\mathstrut#1\,}}
\newcommand{\vectt}[1]{\overrightarrow{\,\mathstrut\text{#1}\,}}% vecteur(AB)
\renewcommand{\theenumi}{\textbf{\arabic{enumi}}}
\renewcommand{\labelenumi}{\textbf{\theenumi.}}
\renewcommand{\theenumii}{\textbf{\alph{enumii}}}
\renewcommand{\labelenumii}{\textbf{\theenumii.}}
\def\Oij{$\left(\text{O}~;~\vect{\imath},~\vect{\jmath}\right)$}
\def\Oijk{$\left(\text{O}~;~\vect{\imath},~\vect{\jmath},~\vect{k}\right)$}
\def\Ouv{$\left(\text{O}~;~\vect{u},~\vect{v}\right)$}
\usepackage{fancyhdr}
\usepackage[dvips]{hyperref}
\hypersetup{%
pdfauthor = {APMEP},
pdfsubject = {Corrigé du baccalauréat S},
pdftitle = {Nouvelle Calédonie - novembre 2018},
allbordercolors = white,
pdfstartview=FitH} 
\usepackage[frenchb]{babel}
\DecimalMathComma
\usepackage[np]{numprint}
\renewcommand{\d}{\mathrm{\,d}}%     le d de différentiation
\newcommand{\e}{\mathrm{\,e\,}}%      le e de l'exponentielle
\renewcommand{\i}{\mathrm{\,i\,}}%    le i des complexes
\newcommand{\ds}{\displaystyle}

\begin{document}

\setlength\parindent{0mm}
\rhead{\textbf{A. P{}. M. E. P{}.}}
\lhead{\small Corrigé du baccalauréat S}
\lfoot{\small{Nouvelle Calédonie}}
\rfoot{\small{27 novembre 2018}}
\addtolength{\headheight}{\baselineskip}
\pagestyle{fancy}
\thispagestyle{empty}
\marginpar{\rotatebox{90}{\textbf{A. P{}.  M. E. P{}.}}}
\begin{center}
{\Large \textbf{\decofourleft~Corrigé du baccalauréat S -- Nouvelle Calédonie~\decofourright\\[7pt]
27 novembre 2018}}
\end{center}

\vspace{0,5cm}

\textbf{Exercice 1 \hfill 6 points} 

\medskip

\textbf{Commun à tous les candidats}

\bigskip

Soient $f$ et $g$ les fonctions définies sur $]0~;~+\infty[$ par
$f(x)=\e^{-x}$ et $g(x) = \dfrac{1}{x^2} \e^{-\frac{1}{x}}$.

On admet que $f$ et $g$ sont dérivables sur $]0~;~+\infty[$. On note $f'$ et $g'$ leurs fonctions dérivées respectives.
Les représentations graphiques de $f$ et $g$ dans un repère orthogonal, nommées respectivement $\mathcal{C}_f$ et $\mathcal{C}_g$ sont données ci-dessous:

\begin{center}
\psset{xunit=1cm,yunit=7cm,comma}
\def\xmin {-0.9}   \def\xmax {10.4}
\def\ymin {-0.18}   \def\ymax {1.15}
\begin{pspicture*}(\xmin,\ymin)(\xmax,\ymax)
\psgrid[xunit=0.5,yunit=0.2,subgriddiv=1, gridlabels=0, gridcolor=lightgray](-2,-1)(21,6)
\psaxes[arrowsize=3pt 3, ticksize=-2pt 2pt, Dx=1,Dy=0.2,labelFontSize=\scriptstyle]{->}(0,0)(0,0)(\xmax,\ymax) 
%\psaxes[ linewidth=1.8pt]{->}(0,0)(1,1)[$\vec\imath$,d][$\vec\jmath$,l]
\def\f{2.7183 x neg exp}
\def\g{2.7183 -1 x div exp x x mul div}
\pscustom[fillstyle=hlines,fillcolor=lightgray]
{\psplot[plotpoints=3000]{0.01}{1}{\f}
\psplot[plotpoints=3000]{1}{0.01}{\g}
\closepath
}
\pscustom[fillstyle=solid,fillcolor=lightgray]
{\psplot[plotpoints=3000]{1}{\xmax}{\g}
\psplot[plotpoints=3000]{\xmax}{1}{\f}
\closepath
}
\psplot[plotpoints=1000,linecolor=red,linestyle=dashed,dash=2pt 2pt]{0.01}{\xmax}{\f}
\uput[ur](0.2,0.82){$\red \mathcal{C}_f$}
\psplot[plotpoints=1000,linecolor=blue]{0.01}{\xmax}{\g}
\uput[ur](2.4,0.1){$\blue \mathcal{C}_g$}
\psdots(1,0.368)(0.5,0.54)
\psline[linestyle=dashed,dash=2pt 2pt](1,0)(1,0.368)
\psline[linestyle=dashed,dash=2pt 2pt]{<->}(-0.4,0.54)(1.4,0.54)
\psline[linestyle=dashed,dash=2pt 2pt](0.5,0.54)(0.5,0) \uput*[d](0.5,0){\footnotesize $0,5$}
\end{pspicture*}
\end{center}

\textbf{Partie A -- Conjectures graphiques}

\medskip

%Dans chacune des questions de cette partie, aucune explication n'est demandée.

\begin{enumerate}
\item D'après le graphique, on peut dire qu'une solution de l'équation $f(x)=g(x)$ sur $]0~;~+\infty[$ est $x=1$.
\item D'après le graphique, on peut dire qu'une solution de l'équation $g'(x)=0$ sur $]0~;~+\infty[$ est $x=0,5$.
\end{enumerate}

\bigskip

\textbf{Partie B -- Étude de la fonction \boldmath{$g$}}

\medskip

\begin{enumerate}
\item On cherche la limite de $g(x)$ quand $x$ tend vers $+\infty$.

$\left .
\begin{array}{l}
\ds\lim_{x\to +\infty} \dfrac{1}{x^2}=0\\[9pt]
\left .
\begin{array}{@{}l}
\ds\lim_{x\to +\infty} -\dfrac{1}{x} = 0\\[9pt]
\ds\lim_{X\to 0} \e^{X} = 1\\
\end{array}
\right \rbrace
\text{ donc }
\ds\lim_{x\to +\infty} \e^{-\frac{1}{x}}=1
\end{array}
\right \rbrace$
donc
$\ds\lim_{x\to +\infty} \dfrac{1}{x^2} \e^{-\frac{1}{x}}=0$

\smallskip

On peut donc dire que $\ds\lim_{x\to +\infty} g(x)=0$.

\item On admet que la fonction $g$ est strictement positive sur $]0~;~+\infty[$.

Soit $h$ la fonction définie sur $]0~;~+\infty[$ par $h(x)=\ln\left ( g(x) \strut\right )$.

\begin{enumerate}
\item Pour tout nombre réel $x$ strictement positif, 

$h(x)=  \ln\left ( g(x) \strut\right ) 
= \ln\left (  \dfrac{1}{x^2} \e^{-\frac{1}{x}}\right )
= \ln\left (\dfrac{1}{x^2}\right ) +  \ln\left ( \e^{-\frac{1}{x}} \right )
= -\ln\left (x^2\strut \right ) -\dfrac{1}{x}
= -2 \ln x  - \dfrac{1}{x}\\[7pt]
\phantom{h(x)}
= \dfrac{-1-2x\ln x}{x}$.

\item On calcule la limite de $h(x)$ quand $x$ tend vers 0.

On sait que $\ds\lim_{\substack{x\to 0\\x>0}} x \ln x = 0$.
On en déduit que \\
$\ds\lim_{\substack{x\to 0\\x>0}} \left (-1-2x \ln x\strut\right ) = -1$ et donc que
$\ds\lim_{\substack{x\to 0\\x>0}} \dfrac{-1-2x\ln x}{x} = -\infty$, c'est-à-dire
$\ds\lim_{\substack{x\to 0\\x>0}} h(x)= -\infty$. 

\item% En déduire la limite de $g(x)$ quand $x$ tend vers 0.
Pour tout $x>0$, $h(x)=\ln\left (g(x)\strut\right )$ donc $g(x) = \e^{h(x)}$.

$\left .
\begin{array}{@{}l}
\ds\lim_{\substack{x\to 0\\x>0}} h(x) = -\infty\\[7pt]
\ds\lim_{X\to -\infty} \e^{X} = 0\\
\end{array}
\right \rbrace$
donc
$\ds\lim_{\substack{x\to 0\\x>0}} \e^{h(x)}=0$
ce qui veut dire que
$\ds\lim_{\substack{x\to 0\\x>0}} g(x)=0$.


\end{enumerate}

\item% Démontrer que, pour tout nombre réel $x$ strictement positif, \[g'(x)= \dfrac{\e^{-\frac{1}{x}}\left (1-2x \strut\right )}{x^4}.\]
Pour tout $x>0$, la dérivée de la fonction $x \mapsto \dfrac{1}{x}$ est $x \mapsto -\dfrac{1}{	x^2}$.\\[5pt]
Pour tout $x>0$, la dérivée de la fonction $x \mapsto \dfrac{1}{x^2}$ est $x \mapsto -\dfrac{-2x}{x^4}$.\\[5pt]
Pour tout $x>0$, la dérivée de la fonction $x \mapsto \e^{-\frac{1}{x}}$ est $x \mapsto \dfrac{1}{x^2}\e^{-\frac{1}{x}}$.

\smallskip

Donc pour tout $x>0$, 
$g'(x) = \left ( -\dfrac{2x}{x^4}\right ) \times \e^{-\frac{1}{x}} + \dfrac{1}{x^2}\times \left ( \dfrac{1}{x^2} \e^{-\frac{1}{x}}\right )
= \dfrac{\e^{-\frac{1}{x}}\left (1-2x \strut\right )}{x^4}$.

\item Sur $]0~;~+\infty[$, $x^4>0$ et $\e^{-\frac{1}{x}}>0$ donc $g'(x)$ est du signe de $1-2x$:

\begin{list}{\textbullet}{}
\item la fonction $g$ est strictement croissante sur $]0~;~0,5]$;
\item la fonction $g$ est strictement décroissante sur $[0,5~;~+\infty[$.
\end{list}

\end{enumerate}

\bigskip

\textbf{Partie C -- Aire des deux domaines compris entre les courbes \boldmath{$\mathcal{C}_f$} et \boldmath{$\mathcal{C}_g$}}

\medskip 

\begin{enumerate}
\item% Démontrer que le point A de coordonnées $\left (1~;~\e^{-1}\strut\right )$ est un point d'intersection de $\mathcal{C}_f$ et $\mathcal{C}_g$.
Soit A le point de coordonnées $\left (1~;~\e^{-1}\strut\right )$.

$f(x_{\text{A}})=f(1) = \e^{-1} = y_{\text{A}}$ donc le point A appartient à la courbe $\mathcal{C}_f$.

$g(x_{\text{A}})=g(1) = \dfrac{1}{1^2}\e^{-\frac{1}{1}} = \e^{-1} = y_{\text{A}}$ donc le point A appartient à la courbe $\mathcal{C}_g$.

Donc le point A est un point d'intersection des courbes $\mathcal{C}_f$ et $\mathcal{C}_g$.

\smallskip

On admet que ce point est l'unique point d'intersection de $\mathcal{C}_f$ et $\mathcal{C}_g$, et que $\mathcal{C}_f$ est au dessus de $\mathcal{C}_g$ sur l'intervalle $]0~;~1[$ et en dessous sur l'intervalle $]1~;~+\infty[$.

\item Soient $a$ et $b$ deux réels strictement positifs. %Démontrer que

%\[\ds\int_{a}^{b} \left ( f(x)-g(x)\strut\right ) \d x = \e^{-a} + \e^{-\frac{1}{a}} - \e^{-b} - \e^{-\frac{1}{b}}.\]

La fonction $f$ définie par $f(x)=\e^{-x}$ a pour primitive la fonction $x \mapsto -\e^{-x}$.

La fonction $g$ est définie par $g(x)= \dfrac{1}{x^2} \e^{-\frac{1}{x}}$ de la forme $u'(x)\e^{u(x)}$ d'après ce qui a été vu précédemment; elle a donc pour primitive la fonction $x \mapsto \e^{u(x)}$ c'est-à-dire $x\mapsto \e^{-\frac{1}{x}}$.

La fonction $\left (f-g\strut\right )$ a donc pour primitive la fonction $x\mapsto -\e^{-x} - \e^{-\frac{1}{x}}$.

On en déduit que

$\ds\int_{a}^{b} \left ( f(x)-g(x)\strut\right ) \d x 
= \left [ -\e^{-x} - \e^{-\frac{1}{x}}\strut \right ]_{a}^{b} 
=  - \e^{-b} - \e^{-\frac{1}{b}} - \left ( -\e^{-a} - \e^{-\frac{1}{a}} \right )\\
\phantom{\ds\int_{a}^{b} \left ( f(x)-g(x)\strut\right ) \d x}
= \e^{-a} + \e^{-\frac{1}{a}} - \e^{-b} - \e^{-\frac{1}{b}}$.

\item %Démontrer que \[\ds\lim_{a\to 0} \ds\int_{a}^{1} \left ( f(x)-g(x)\strut\right ) \d x =1-2\e^{-1}.\]
D'après la question précédente,

$\ds\int_{a}^{1} \left ( f(x)-g(x)\strut\right ) \d x  = \e^{-a} + \e^{-\frac{1}{a}} - \e^{-1} -\e^{\frac{-1}{1}}
= \e^{-a} + \e^{-\frac{1}{a}} - 2\e^{-1}$.

$\ds\lim_{a\to 0} \e^{-a} = \e^{0}=1$

$\left .
\begin{array}{@{}l}
\ds\lim_{\substack{a\to 0\\a>0}} -\dfrac{1}{a} = -\infty\\[7pt]
\ds\lim_{X\to -\infty} \e^{X} = 0\\
\end{array}
\right \rbrace$ donc
$\ds\lim_{\substack{a\to 0\\a>0}} \e^{-\frac{1}{a}} = 0$

On peut donc déduire que 
$\ds\lim_{\substack{a\to 0\\a>0}}  \e^{-a} + \e^{-\frac{1}{a}} = 1$
et donc que
$\ds\lim_{a\to 0} \ds\int_{a}^{1} \left ( f(x)-g(x)\strut\right ) \d x =1-2\e^{-1}$.

\item On admet que
$\ds\lim_{a\to 0} \ds\int_{a}^{1} \left ( f(x)-g(x)\strut\right ) \d x =\ds\lim_{b\to +\infty} \ds\int_{1}^{b} \left ( g(x)-f(x)\strut\right ) \d x .$

%Interpréter graphiquement cette égalité.

Sur l'intervalle $]0~;~1[$, la courbe $\mathcal{C}_f$ est au dessus de la courbe $\mathcal{C}_g$ donc\\
$\ds\lim_{a\to 0} \ds\int_{a}^{1} \left ( f(x)-g(x)\strut\right ) \d x$ représente l'aire de la partie du plan comprise entre les courbes $\mathcal{C}_f$ et $\mathcal{C}_g$, et les droites d'équation $x=0$ et $x=1$. \\
C'est l'aire de la région hachurée sur le graphique.

Sur l'intervalle $]1~;~+\infty[$, la courbe $\mathcal{C}_g$ est au dessus de la courbe $\mathcal{C}_f$ donc\\
$\ds\lim_{b\to +\infty} \ds\int_{1}^{b} \left ( g(x)-f(x)\strut\right ) \d x $ représente l'aire de la partie du plan comprise entre les courbes $\mathcal{C}_g$ et $\mathcal{C}_f$, et les droites $x=1$ et $x=b$ quand $b$ tend vers $+\infty$. \\
C'est l'aire de la région grisée sur le graphique.

On peut donc dire que ces deux aires sont égales. 
 
\end{enumerate}

\vspace{0,5cm}

\textbf{Exercice 2 \hfill 3 points}

\medskip

\textbf{Commun à tous les candidats}

\bigskip

Une épreuve de culture générale consiste en un questionnaire à choix multiple (QCM) de vingt questions. Pour chacune d'entre elles, le sujet propose quatre réponses possibles, dont une seule est correcte. À chaque question, le candidat ou la candidate doit nécessairement choisir une seule réponse. Cette personne gagne un point par réponse correcte et ne perd auxun point si sa réponse est fausse.

\medskip

\begin{list}{\textbullet}{On considère trois candidats:}
\item Anselme répond complètement au hasard à chacune des vingt questions.

Autrement dit, pour chacune des questions, la probabilité qu'il réponde correctement est égale à $\dfrac{1}{4}$;
\item Barbara est un peu mieux préparée. On considère que pour chacune des vingt questions, la probabilité qu'elle réponde correctement est de $\dfrac{1}{2}$;
\item Camille fait encore mieux: pour chacune des questions, la probabilité qu'elle réponde correctement est de $\dfrac{2}{3}$.
\end{list}

\medskip

\begin{enumerate}
\item On note $X$, $Y$ et $Z$ les variables aléatoires égales aux notes respectivement obtenues par Anselme, Barbara et Camille.

\begin{enumerate}
\item %Quelle est la loi de probabilité suivie par la variable aléatoire $X$? Justifier.
Anselme répond au hasard à chaque question donc il a une probabilité de répondre juste à une question de $p=\dfrac{1}{4}$.

Il y a 20 questions qui sont indépendantes donc la variable aléatoire $X$ qui donne le nombre de bonnes réponses d'Anselme, donc sa note, suit la loi binomiale de paramètres $n=20$ et $p=0,25$.

\item À l'aide de la calculatrice, l'arrondi au millième de la probabilité $P(X \geqslant 10)$ est $0,014$.
\end{enumerate}
\end{enumerate}

Dans la suite, on admettra que $P(Y\geqslant 10) \approx 0,588$ et $P(Z\geqslant 10)\approx 0,962$.

\begin{enumerate}
\setcounter{enumi}{1}
\item On choisit au hasard la copie d'un de ces trois candidats.

\begin{list}{\textbullet}{On note $A$, $B$, $C$ et $M$ les événements:}
\item $A$: \og la copie choisie est celle d'Anselme \fg{};
\item $B$: \og la copie choisie est celle de Barbara \fg{};
\item $C$: \og la copie choisie est celle de Camille \fg{};
\item $M$: \og la copie choisie obtient une note supérieure ou égale à 10 \fg{}. 
\end{list}

\smallskip

On constate, après l'avoir corrigée, que la copie choisie obtient une note supérieure ou égale à 10 sur 20.

%\smallskip

La probabilité qu'il s'agisse de la copie de Barbara est $P_{M}(B)$ soit $\dfrac{P\left (B\cap M\right )}{P(M)}$.

\begin{list}{\textbullet}{}
\item On choisit au hasard la copie d'un des trois candidats donc 
$P(A)=P(B)=P(C)=\dfrac{1}{3}$.

\item $P(B \cap M) = P(B) \times P_{B}(M)$. \\
D'après le contexte, $P_{B}(M) = P(Y\geqslant 10)$ donc $P(B \cap M) = \dfrac{1}{3}\times 0,588$.

\item D'après la formule des probabilités totales

$P(M)= P(A\cap M) + P(B\cap M) + P(C\cap M) = P(A)\times P_{A}(M) + P(B)\times P_{B}(M) + P(C)\times P_{C}(M) \\[5pt]
\phantom{P(M)}
= P(A)\times P(X\geqslant 10) + P(B)\times P(Y\geqslant 10) + P(C)\times P(Z\geqslant 10)\\[5pt]
\phantom{P(M)}
= \dfrac{1}{3} \times 0,014 + \dfrac{1}{3}\times 0,588 + \dfrac{1}{3}\times 0,962
= \dfrac{1,564}{3}$ 
\end{list}

Donc $P_{M}(B) = \dfrac{\frac{0.588}{3}}{\frac{1,564}{3}}$ dont l'arrondi au millième est $0,376$.
%On donnera l'arrondi au millième de cette probabilité.

\end{enumerate}

\vspace{0,5cm}

\textbf{Exercice 3 \hfill 6 points}

\medskip

\textbf{Commun à tous les candidats}

\bigskip

Soit ABCDEFGH le cube représenté ci-dessous.

\begin{list}{\textbullet}{On considère:}
\item I et J les milieux respectifs des segments [AD] et [BC];
\item P le centre de la face ABFE, c'est-à-dire l'intersection des diagonales (AF) et (BE);
\item Q le milieu du segment [FG].
\end{list}


\begin{center}
\scalebox{0.8}
{
\psset{unit=0.8cm,radius=2pt}
\def\xmin {-3}   \def\xmax {11}
\def\ymin {-3}   \def\ymax {10}
\begin{pspicture*}(\xmin,\ymin)(\xmax,\ymax)
%\psgrid[subgriddiv=1, gridlabels=0, gridcolor=lightgray] 
\Cnode*(0,0){A} \Cnode*(5,-1){B} \Cnode*(4,2){D} \Cnode*(9,1){C}
\Cnode*(0,6){E} \Cnode*(5,5){F} \Cnode*(4,8){H} \Cnode*(9,7){G}
\psline(E)(A)(B)(F)(E)(H)(G)(F)
\psline(B)(C)(G)
\psline[linestyle=dashed](A)(D)(H) \psline[linestyle=dashed](D)(C)
\uput[dl](A){A} \uput[d](B){B} \uput[dr](C){C} \uput[ur](D){D}
\uput[ul](E){E} \uput[u](F){F} \uput[10](G){G} \uput[u](H){H}
\Cnode*(2,1){I} \Cnode*(7,0){J} \Cnode*(2.5,2.5){P} \Cnode*(7,6){Q}
\uput[d](I){I} \uput[d](J){J} \uput[u](P){P} \uput[ul](Q){Q}
\psline[linestyle=dashed](P)(Q) \psline(P)(-2,-1) \psline(Q)(11.5,9.5)
\psline[linestyle=dashed](J)(0,1.4) \psline(J)(12,-1) \psline(0,1.4)(-3,2)
\psset{linewidth=2pt}
\psline[linestyle=dashed]{->}(A)(I) \psline{->}(A)(0,3) \psline{->}(A)(2.5,-0.5)
\end{pspicture*}
}
\end{center}

On se place dans le repère orthonormé $\left ( \text{A}~;~\frac{1}{2}\vectt{AB}\;,\;\frac{1}{2}\vectt{AD}\;,\;\frac{1}{2}\vectt{AE}\right )$.

Dans tout l'exercice, on pourra utiliser les coordonnées des points de la figure sans les justifier.

Ces points ont pour coordonnées:
%\newcommand{\coo}[3]{\;\begin{pmatrix} #1\\ #2 \\ #3 \end{pmatrix}}
\newcommand{\coo}[3]{\left( #1~;~#2~;~#3\strut\right ) }

\hfill{}A$\coo{0}{0}{0}$; B$\coo{2}{0}{0}$; D$\coo{0}{2}{0}$; E$\coo{0}{0}{2}$;
C$\coo{2}{2}{0}$; F$\coo{2}{0}{2}$; H$\coo{0}{2}{2}$; G$\coo{2}{2}{2}$;
I$\coo{0}{1}{0}$; J$\coo{2}{1}{0}$; P$\coo{1}{0}{1}$ et Q$\coo{2}{1}{2}$\hfill{}

\smallskip

On admet qu'une représentation paramétrique de la droite (IJ) est
$\left \lbrace
\begin{array}{l !{=} l}
x& r\\
y & 1\\
z & 0\\
\end{array}
\right . , \quad r\in\R$

\begin{enumerate}
\item% Vérifier qu'une représentation paramétrique de la droite (PQ) est
%
%\[\left \lbrace
%\begin{array}{l !{=} l}
%x& 1+t\\
%y & t \\
%z & 1+t\\
%\end{array}
%\right . ,\quad t\in\R\]
La droite (PQ) est l'ensemble des points M$\coo{x}{y}{z}$ tels que les vecteurs $\vectt{PM}$ et $\vectt{PQ}$ soient colinéaires, c'est-à-dire tels que $\vectt{PM} = t.\vectt{PQ}$ avec $t\in\R$.

$\vectt{PM}$ a pour coordonnées $\coo{x-1}{y}{z-1}$ et $\vectt{PQ}$ a pour coordonnées $\coo{2-1}{1-0}{2-1}=\coo{1}{1}{1}$

$\vectt{PM} = t.\vectt{PQ}
\iff
\left \lbrace
\begin{array}{l !{=} l}
x-1& t\times 1\\
y & t\times 1 \\
z-1 & t\times 1\\
\end{array}
\right .
\iff
\left \lbrace
\begin{array}{l !{=} l}
x& 1+ t\\
y & t \\
z & 1+t\\
\end{array}
\right .$

La droite (PQ) a pour représentation paramétrique
$\left \lbrace
\begin{array}{l !{=} l}
x& 1+t\\
y & t \\
z & 1+t\\
\end{array}
\right . ,\quad t\in\R$.

\end{enumerate}

Soient $t$ un nombre réel et M\,$(1+t~;~t~;~1+t)$ le point de la droite (PQ) de paramètre $t$.

\begin{enumerate}
\setcounter{enumi}{1}
\item 
\begin{enumerate}
\item On admet qu'il existe un unique point K appartenant à la droite (IJ) tel que (MK) soit orthogonale à (IJ).

%Démontrer que les coordonnées de ce point K sont $(1+t~;~1~;~0)$.

Le point K appartient à la droite (IJ) dont on connait une représentation paramétrique donc les coordonnées de K sont de la forme $\coo{r}{1}{0}$.

Les droites (IJ) et (MK) sont orthogonales donc les vecteurs $\vectt{IJ}$ et $\vectt{MK}$ sont orthogonaux; leur produit scalaire est donc nul.

$\vectt{IJ}$ a pour coordonnées $\coo{2-0}{1-1}{0-0} = \coo{2}{0}{0}$, et $\vectt{MK}$ a pour coordonnées $\coo{x_{\text K} - x_{\text M} }{y_{\text K} - y_{\text M}}{z_{\text K} - z_{\text M}}=\coo{r-1-t}{1-t}{-1-t}$.

$\vectt{IJ}.\vectt{MK}=0
\iff
2(r-1-t)+0(1-t) + 0(-1-t) = 0 \iff r=1+t$

Le point K a donc pour coordonnées $\coo{1+t}{1}{0}$.

\item% En déduire que $\text{MK} = \ds\sqrt{2+2t^2}$.
D'après la question précédente, le vecteur $\vectt{MK}$ a pour coordonnées 
$\coo{0}{1-t}{-1-t}$ donc
$\text{MK} = \ds\sqrt{0^2 + (1-t)^2+ (-1-t)^2}
= \ds\sqrt{1-2t+t^2+1+2t+t^2} = \ds\sqrt{2+2t^2}$.

\end{enumerate}

\item
\begin{enumerate}
\item Les trois points H, G et B ne sont pas alignés donc ils définissent le plan (HGB).

Soit $\mathcal{P}$ le plan d'équation $y-z=0$.

$y_{\text H} -z_{\text H} = 2-2=0$ donc le point H appartient au plan $\mathcal{P}$.

$y_{\text G} -z_{\text G} = 2-2=0$ donc le point G appartient au plan $\mathcal{P}$.

$y_{\text B} -z_{\text B} = 0-0=0$ donc le point B appartient au plan $\mathcal{P}$.

Le plan $\mathcal{P}$ est donc le plan (HGB) ce qui veut dire que le plan (HGB) a pour équation cartésienne $y-z=0$.

\item On admet qu'il existe un unique point L appartenant au plan (HGB) tel que (ML) soit orthogonale à (HGB).

\smallskip

On suppose que  le point L a pour coordonnées $\left (1+t~;~\dfrac{1}{2}+t~;~\dfrac{1}{2}+t\right )$.

\begin{list}{\textbullet}{}
\item $y_{\text L} - z_{\text L} = \dfrac{1}{2}+t - \left ( \dfrac{1}{2}+t\right ) = 0$ donc $\text L \in \text{(HGB)}$.
\item Le vecteur $\vectt{ML}$ a pour coordonnées\\ 
$\coo{1+t -\left (1+t \right )}{\dfrac{1}{2}+t-t}{\dfrac{1}{2}+t- \left ( 1+t\right )}=\coo{0}{\dfrac{1}{2}}{-\dfrac{1}{2}}$.

\item Le vecteur $\vectt{HG}$ a pour coordonnées $\coo{2-0}{2-2}{2-2} = \coo{2}{0}{0}$.

$\vectt{ML}.\vectt{HG} = 0\times 2 + \dfrac{1}{2}\times 0 + \left (-\dfrac{1}{2}\right )\times 0 = 0$ donc $\vectt{ML}\perp \vectt{HG}$.
 
\item Le vecteur $\vectt{HB}$ a pour coordonnées $\coo{2-0}{0-2}{0-2} = \coo{2}{-2}{-2}$.

$\vectt{ML}.\vectt{HB} = 0\times 2 + \dfrac{1}{2}\times (-2) + \left (-\dfrac{1}{2}\right )\times 2 = 0-1+1 = 0$ donc $\vectt{ML}\perp \vectt{HB}$.

\end{list}

Le vecteur $\vectt{ML}$ est orthogonal à deux vecteurs $\vectt{HG}$ et $\vectt{HB}$ non colinéaires, donc le  vecteur $\vectt{ML}$ est orthogonal au plan (HGB).

Si L a pour coordonnées $\left (1+t~;~\dfrac{1}{2}+t~;~\dfrac{1}{2}+t\right )$, alors L appartient au plan (HGB) et la droite (ML) est orthogonale au plan (HGB).

\item %En déduire que la distance ML est indépendante de $t$.
Le vecteur $\vectt{ML}$ a pour coordonnées $\coo{0}{\dfrac{1}{2}}{-\dfrac{1}{2}}$; ces coordonnées ne dépendent pas de $t$ donc la distance ML ne dépend pas de $t$. 

$\text{ML} = \ds\sqrt{0^2 + \left (\dfrac{1}{2}\right )^2 + \left ( -\dfrac{1}{2}\right )^2}
= \ds\sqrt{\dfrac{1}{4}+\dfrac{1}{4}}= \ds\sqrt{\dfrac{1}{2}} = \ds\dfrac{\sqrt{2}}{2}$

\end{enumerate}

\item %Existe-t-il une valeur de $t$ pour laquelle la distance MK est égale à la distance ML?
La distance MK est égale à ML si et seulement si
$\ds\sqrt{2+2t^2} = \ds\sqrt{\dfrac{1}{2}}$
ce qui équivaut à
$2+2t^2 = \dfrac{1}{2}$ ou encore $2t^2 = -\dfrac{3}{2}$.

L'équation $2t^2 = -\dfrac{3}{2\rule[-3pt]{0pt}{0pt}}$ n'a pas de solution donc il n'existe pas de valeur de $t$ pour laquelle les distances MK et ML sont égales.


\end{enumerate}

\vspace{0,5cm}

\textbf{Exercice 4 \hfill 5 points}

\medskip

\textbf{Candidats n'ayant pas suivi l'enseignement de spécialité}

\bigskip

On définit la suite de nombres complexes $(z_n)$ de la manière suivante: $z_0=1$ et, pour tout entier naturel $n$,
$z_{n+1} = \dfrac{1}{3} z_{n} + \dfrac{2}{3}\i.$

On se place dans un plan muni d'un repère orthonormé direct \Ouv.

Pour tout entier naturel $n$, on note A$_{n}$ le point du plan d'affixe $z_n$.

Pour tout entier naturel $n$, on pose $u_n=z_n-\i$ et on note B$_n$ le point d'affixe $u_n$.

On note C le point d'affixe $\i$.

\begin{enumerate}
\item De $u_n=z_n-\i$, on  déduit que $z_n=u_n+\i$.

%Exprimer $u_{n+1}$ en fonction de $u_n$, pour tout entier naturel $n$.

$u_{n+1}= z_{n+1} -\i = \dfrac{1}{3} z_n +\dfrac{2}{3}\i - \i 
= \dfrac{1}{3} \left ( u_n+\i \strut\right ) -\dfrac{1}{3}\i
= \dfrac{1}{3} u_n + \dfrac{1}{3}\i - \dfrac{1}{3}\i
= \dfrac{1}{3} u_n$ 

\item% Démontrer que,  pour tout entier naturel $n$, $u_n=\left (\dfrac{1}{3}\right )^{n} \left (1-\i\right )$. 
La suite $(u_n)$ est géométrique de raison $q=\dfrac{1}{3}$ et de premier terme $u_0=v_0 -\i = 1-\i$.

On a donc, pour tout $n$, $u_n=u_0\times q^n = \left (1-\i\strut\right ) \left (\dfrac{1}{3} \right )^n$.


\item 
\begin{enumerate}
\item Pour tout entier naturel $n$, \\
$\left |u_n\strut\right | = \left | \left (\dfrac{1}{3}\right )^n\left (1-\i\right )\right | = \left (\dfrac{1}{3}\right )^n \left |1-\i\strut\right |
= \left (\dfrac{1}{3}\right )^n \ds\sqrt{1^2 + (-1)^2}
= \ds\sqrt{2}\left (\dfrac{1}{3}\right )^n$.

\item %Démontrer que \[\lim_{n\to +\infty} \left |z_n-\i \strut\right |=0.\]
$\left |z_n-\i \strut\right | = \left |u_n \strut\right | = \ds\sqrt{2}\left (\dfrac{1}{3}\right )^n$

Or $-1<\dfrac{1}{3} < \dfrac{1}{3}$, donc $\ds\lim_{n \to +\infty} \left (\dfrac{1}{3} \right )^n =0$ donc 
$\ds\lim_{n\to +\infty} \left |z_n-\i \strut\right |=0$.

\item% Quelle interprétation géométrique peut-on donner de ce résultat?
$z_n$ est l'affixe du point A$_n$, et $\i$ est l'affixe du point C; donc $\left |z_n-\i \strut\right |=\text{A}_n\text{C}$.

$\ds\lim_{n\to +\infty} \left |z_n-\i \strut\right |=0$ signifie que $\ds\lim_{n\to +\infty} \text{A}_n\text{C}=0$ ce qui veut dire que, lorsque $n$ tend vers $+\infty$, le point A$_n$ tend vers le point C.

\end{enumerate}

\item 
\begin{enumerate}
\item Soit $n$ un entier naturel.% Déterminer un argument de $u_n$.

$u_n=\left (\dfrac{1}{3}\right )^n\left (1-\i\strut\right )$ donc $\arg(u_n)=\arg\left (1-\i\strut\right )$

$1-\i =\ds\sqrt{2}\left (\dfrac{\ds\sqrt{2}}{2} - \i\dfrac{\ds\sqrt{2}}{2} \right ) =
\ds\sqrt{2} \left ( \cos\left (-\dfrac{\pi}{4}\right ) + \i \sin\left (-\dfrac{\pi}{4}\right )\right )$ donc 
$1-\i$ a pour argument $ -\dfrac{\pi}{4}$.
 
On en déduit que $\arg(u_n)=-\dfrac{\pi}{4} + k2\pi$ avec $k$ entier relatif.

\item% Démontrer que, lorsque $n$ décrit l'ensemble des entiers naturels, les points B$_n$ sont alignés.
Les points B$_n$ ont pour affixes les nombres $u_n$ dont l'argument est constant à $2\pi$ près. Pour tout $n$, chaque point B$_n$ appartient à la droite d'équation $y=-x$; donc tous les points B$_n$ sont alignés.

\item% Démontrer que, pour tout entier naturel $n$, le point A$_n$ appartient à la droite d'équation réduite: \[y=-x+1.\]

$z_n=u_n+\i = \left (\dfrac{1}{3}\right )^n \left (1-\i\strut\right ) + \i
= \left (\dfrac{1}{3}\right )^n + \left ( 1-\left( \dfrac{1}{3}\right )^n\right  ) \i$

Soit $x$ la partie réelle de $z_n$ et $y$ sa partie imaginaire.

$x=\left (\dfrac{1}{3}\right )^n$ et $y=1-\left (\dfrac{1}{3}\right )^n$ donc $y=1-x$.

On en déduit que $y=-x+1$ et donc que le point A$_n$ d'affixe $z_n$ appartient à la droite d'équation $y=-x+1$.

\end{enumerate}
\end{enumerate}

\vspace{0,5cm}

\textbf{Exercice 4 \hfill 5 points}

\medskip

\textbf{Candidats ayant suivi l'enseignement de spécialité}

\bigskip

On appelle suite de Fibonacci la suite $(u_n)$ définie par $u_0=0$, $u_1=1$ et, pour tout entier naturel $n$,
$u_{n+2} = u_{n+1} + u_{n}.$

On admet que, pour tout entier naturel $n$, $u_n$ est un entier naturel.

%\smallskip
%
%\emph{Les parties \emph{A} et \emph{B} peuvent être traitées de façon indépendante.}

\bigskip

\textbf{Partie A}

\medskip

\begin{enumerate}
\item 
\begin{enumerate}
\item On calcule les termes de la suite de Fibonacci jusqu'à $u_{10}$:

\begin{center}
{\footnotesize
\newcommand{\ca}{\centering\arraybackslash}
\begin{tabular}{|c|*{9}{>{\ca}m{0.6cm}|}*{2}{>{\ca}m{0.75cm}|}}
\hline
$n$ &0 & 1 & 2 & 3 & 4 & 5 & 6 & 7 & 8 & 9 & 10\\
\hline
$u_n$ & 0 & 1 & {\blue $0+1\newline = 1$} & {\blue $1+1\newline = 2$} & {\blue $1+2\newline = 3$} & {\blue $2+3\newline = 5$} & {\blue $3+5\newline = 8$}  & {\blue $5+8\newline = 13$} & {\blue $8+13\newline = 21$} & {\blue $13+21\newline = 34$} & {\blue $21+34\newline = 55$} \\
\hline
\end{tabular}
}
\end{center}

\item %Que peut-on conjecturer sur le PGCD de $u_{n}$ et $u_{n+1}$ pour tout entier naturel $n$?
$\text{PGCD}(u_0,u_1) = \text{PGCD}(0,1) = 1$
\hfill
$\text{PGCD}(u_1,u_2) = \text{PGCD}(1,1) = 1$

$\text{PGCD}(u_2,u_3) = \text{PGCD}(1,2) = 1$
\hfill
$\text{PGCD}(u_3,u_4) = \text{PGCD}(2,3) = 1$

$\text{PGCD}(u_4,u_5) = \text{PGCD}(3,5) = 1$
\hfill
$\text{PGCD}(u_5,u_6) = \text{PGCD}(5,8) = 1$

$\text{PGCD}(u_6,u_7) = \text{PGCD}(8,13) = 1$ car 13 est un nombre premier.

$\text{PGCD}(u_7,u_8) = \text{PGCD}(13,21) = 1$ car 13 est un nombre premier.

$\text{PGCD}(u_8,u_9) = \text{PGCD}(21,34) = 1$ car $21=3\times 7$ et $34=2\times 17$.

$\text{PGCD}(u_9,u_{10}) = \text{PGCD}(34,55) = 1$  $34=2\times 17$ et $55=5\times 11$.

On peut conjecturer que le PGCD de $u_{n}$ et $u_{n+1}$ est 1.
\end{enumerate}

\item On définit la suite $(v_n)$ par $v_n=u_n^2 - u_{n+1}\times u_{n-1}$ pour tout entier naturel $n$ non nul.
\begin{enumerate}
\item On utilise dans ce calcul les égalités:
$u_{n+2} = u_{n+1} + u_n$ et
$u_{n+1} = u_{n} + u_{n-1}$ de laquelle on déduit $u_{n+1} - u_n = u_{n-1}$.

Pour tout entier naturel $n$ non nul:

$v_{n+1}= u_{n+1}^2 - u_{n+2}\times u_{n} 
= u_{n+1}^2 - \left (u_{n+1} + u_{n}\right )\times u_{n} 
= u_{n+1}^2 - u_{n+1}\times u_{n} - u_{n}^2\\[3pt]
\phantom{v_{n+1}}
= u_{n+1} \left ( u_{n+1} - u_n\right )  - u_n^2
= u_{n+1}\times u_{n-1} - u_n^2
= - \left (  u_n^2 - u_{n+1}\times u_{n-1}\right )
= - v_n$

\item% En déduire que, pour tout entier naturel $n$ non nul, $u_n^2 - u_{n+1}\times u_{n-1} = \left (-1\right )^{n-1}.$

On déduit de la question précédente que la suite $(v_n)$ est géométrique de raison $q=-1$ et de premier terme $v_1=u_1^2 - u_2\times u_0=1^2-1\times 0 = 1$.

Donc pour tout entier naturel $n$ non nul, $v_n=v_1\times q^{n-1}= \left (-1\right )^{n-1}$.

Et donc pour tout entier naturel $n$ non nul, $u_n^2 - u_{n+1}\times u_{n-1} = \left (-1\right )^{n-1}.$

\item% Démontrer alors la conjecture émise à la question \textbf{1.b.}
Pour tout entier naturel $n$ non nul, on appelle $d$ le PGCD de $u_{n}$ et $u_{n+1}$.

Alors $d$ divise toute combinaison linéaire de $u_{n}$ et $u_{n+1}$ donc $d$ divise $u_n\times u_n - u_{n+1}\times u_{n-1} $; on en déduit que $d$ divise $(-1)^{n-1}$ et comme le PGCD est un nombre positif, on peut déduire que $d=1$.

Pour $n=0$, on a vu que le PGCD de $u_0$ et $u_1$ était 1.

On peut donc dire que, pour tout entier naturel $n$, le PGCD de $u_n$ et $u_{n+1}$ est 1.

\end{enumerate}
\end{enumerate}

\bigskip

\textbf{Partie B}

\medskip

On considère la matrice
$F=\begin{pmatrix} 1 & 1 \\ 1 & 0\end{pmatrix}$.

\begin{enumerate}
\item À la calculatrice, on trouve 
$F^2=\begin{pmatrix} 2 & 1 \\ 1 & 1\end{pmatrix}$ et 
$F^3=\begin{pmatrix} 3 & 2 \\ 2 & 1\end{pmatrix}$.% On pourra utiliser la calculatrice.


\item Soit $\mathcal{P}_n$ la propriété $F^n = \begin{pmatrix} u_{n+1} & u_n \\ u_n & u_{n-1} \end{pmatrix}$.

\begin{list}{\textbullet}{}
\item Pour $n=1$, $F^{n}=F^{1}=F = \begin{pmatrix} 1 & 1 \\ 1 & 0\end{pmatrix}
= \begin{pmatrix} u_2 & u_1 \\ u_1 & u_0\end{pmatrix}$ d'après la question \textbf{A.1.a.}

Donc la propriété est vraie au rang $1$

\item Soit $k$ un entier naturel non nul tel que la propriété soit vraie au rang $k$; on a donc
$F^k = \begin{pmatrix} u_{k+1} & u_k \\ u_k & u_{k-1} \end{pmatrix}$ (hypothèse de récurrence).

$F^{k+1} = F^k \times F 
= \begin{pmatrix} u_{k+1} & u_k \\ u_k & u_{k-1} \end{pmatrix}
\times \begin{pmatrix} 1 & 1 \\ 1 & 0\end{pmatrix}
= 
\begin{pmatrix} u_{k+1} + u_k & u_{k+1} +0 \\ u_k+u_{k-1} & u_k+0\end{pmatrix}
=
\begin{pmatrix} u_{k+2} & u_{k+1}  \\ u_{k+1} & u_k\end{pmatrix}
$

Donc la propriété est vraie au rang $k+1$.

\item La propriété est vraie au rang 1 et elle est héréditaire pour tout $k \geqslant 1$; d'après le principe de récurrence, la propriété est vraie pour tout $n\geqslant 1$.

\end{list}

On a donc démontré que, pour tout entier naturel $n$ non nul, $F^n = \begin{pmatrix} u_{n+1} & u_n \\ u_n & u_{n-1} \end{pmatrix}$.

\item 
\begin{enumerate}
\item Soit $n$ un entier naturel non nul. %En remarquant que $F^{2n+2} = F^{n+2}\times F^{n}$, démontrer que \[u_{2n+2} =u_{n+2}\times u_{n+1}+ u_{n+1}\times u_n.\]

$F^{2n+2} = F^{n+2}\times F^{n}$ équivaut à
$\begin{pmatrix} u_{2n+3} & u_{2n+2} \\ u_{2n+2} & u_{2n+1} \end{pmatrix}
=
\begin{pmatrix} u_{n+3} & u_{n+2} \\ u_{n+2} & u_{n+1} \end{pmatrix}
\times
\begin{pmatrix} u_{n+1} & u_n \\ u_n & u_{n-1} \end{pmatrix} \iff$

$\begin{pmatrix} u_{2n+3} & {u_{2n+2}} \\ {\blue u_{2n+2}} & u_{2n+1} \end{pmatrix}
=
\begin{pmatrix} u_{n+3}\times u_{n+1}+u_{n+2}\times u_n &  u_{n+3}\times u_n + u_{n+2}\times u_{n-1}  \\ {\blue u_{n+2}\times u_{n+1} + u_{n+1}\times u_n} & u_{n+2}\times u_n + u_{n+1}\times u_{n-1} \end{pmatrix}$

En identifiant les termes situés sur la 1\iere{} colonne 2\ieme{} ligne, on a

\hfill{}$u_{2n+2} = u_{n+2}\times u_{n+1} + u_{n+1}\times u_n$.\hfill{}

\item% En déduire que, pour tout entier naturel $n$ non nul, \[u_{2n+2}^{\phantom{2}} = u_{n+2}^2 - u_n^2.\]

On sait que $u_{n+2}=u_{n+1}+u_n$ donc $u_{n+1}=u_{n+2} -u_n$.

De $u_{2n+2} = u_{n+2}\times u_{n+1} + u_{n+1}\times u_n$
on déduit donc que

$u_{2n+2} = u_{n+2}\left ( u_{n+2} -u_n\right ) + \left ( u_{n+2} -u_n\right ) u_n$
donc que 

$u_{2n+2} = u_{n+2}^2  - u_{n+2}\times u_n +  u_{n+2}\times u_n -u_n^2$.

On a donc démontré que, pour tout entier naturel $n$ non nul, 
$u_{2n+2} = u_{n+2}^2  - u_n^2$.

\end{enumerate}

\item On donne $u_{12}=144$.

%Démontrer en utilisant la question \textbf{3.} qu'il existe un triangle rectangle dont les longueurs des côtés sont toutes des nombres entiers, l'une étant égale à 12.
%
%Donner la longueur des deux autres côtés.

$u_{2n+2} = u_{n+2}^2  - u_n^2$; pour $n=5$, $u_{12} = u_7^2 -u_5^2$ donc $u_5^2+u_{12} = u_7^2$.

$u_{12} = 144 = 12^2$; $u_5=5$ et $u_7=13$

On a donc $5^2+12^2=13^2$.

D'après la réciproque du théorème de Pythagore, on peut affirmer qu'il existe un triangle rectangle dont les longueurs des côtés sont 5, 12 et 13. 
\end{enumerate}
\end{document}